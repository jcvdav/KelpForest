\documentclass[12pt,]{article}
\usepackage{lmodern}
\usepackage{amssymb,amsmath}
\usepackage{ifxetex,ifluatex}
\usepackage{fixltx2e} % provides \textsubscript
\ifnum 0\ifxetex 1\fi\ifluatex 1\fi=0 % if pdftex
  \usepackage[T1]{fontenc}
  \usepackage[utf8]{inputenc}
\else % if luatex or xelatex
  \ifxetex
    \usepackage{mathspec}
  \else
    \usepackage{fontspec}
  \fi
  \defaultfontfeatures{Ligatures=TeX,Scale=MatchLowercase}
\fi
% use upquote if available, for straight quotes in verbatim environments
\IfFileExists{upquote.sty}{\usepackage{upquote}}{}
% use microtype if available
\IfFileExists{microtype.sty}{%
\usepackage{microtype}
\UseMicrotypeSet[protrusion]{basicmath} % disable protrusion for tt fonts
}{}
\usepackage[margin=1in]{geometry}
\usepackage{hyperref}
\hypersetup{unicode=true,
            pdfborder={0 0 0},
            breaklinks=true}
\urlstyle{same}  % don't use monospace font for urls
\usepackage{graphicx,grffile}
\makeatletter
\def\maxwidth{\ifdim\Gin@nat@width>\linewidth\linewidth\else\Gin@nat@width\fi}
\def\maxheight{\ifdim\Gin@nat@height>\textheight\textheight\else\Gin@nat@height\fi}
\makeatother
% Scale images if necessary, so that they will not overflow the page
% margins by default, and it is still possible to overwrite the defaults
% using explicit options in \includegraphics[width, height, ...]{}
\setkeys{Gin}{width=\maxwidth,height=\maxheight,keepaspectratio}
\IfFileExists{parskip.sty}{%
\usepackage{parskip}
}{% else
\setlength{\parindent}{0pt}
\setlength{\parskip}{6pt plus 2pt minus 1pt}
}
\setlength{\emergencystretch}{3em}  % prevent overfull lines
\providecommand{\tightlist}{%
  \setlength{\itemsep}{0pt}\setlength{\parskip}{0pt}}
\setcounter{secnumdepth}{0}
% Redefines (sub)paragraphs to behave more like sections
\ifx\paragraph\undefined\else
\let\oldparagraph\paragraph
\renewcommand{\paragraph}[1]{\oldparagraph{#1}\mbox{}}
\fi
\ifx\subparagraph\undefined\else
\let\oldsubparagraph\subparagraph
\renewcommand{\subparagraph}[1]{\oldsubparagraph{#1}\mbox{}}
\fi

%%% Use protect on footnotes to avoid problems with footnotes in titles
\let\rmarkdownfootnote\footnote%
\def\footnote{\protect\rmarkdownfootnote}

%%% Change title format to be more compact
\usepackage{titling}

% Create subtitle command for use in maketitle
\newcommand{\subtitle}[1]{
  \posttitle{
    \begin{center}\large#1\end{center}
    }
}

\setlength{\droptitle}{-2em}
  \title{\Large{Fished species exhibit latitudinal patterns in size structures along Baja California Peninsula}}
  \pretitle{\vspace{\droptitle}\centering\huge}
  \posttitle{\par}
  \author{}
  \preauthor{}\postauthor{}
  \date{}
  \predate{}\postdate{}

\usepackage{booktabs}
\usepackage{longtable}
\usepackage{array}
\usepackage{multirow}
\usepackage[table]{xcolor}
\usepackage{wrapfig}
\usepackage{setspace}
\doublespacing
\usepackage{lineno}
\linenumbers
\usepackage{pdflscape}

\begin{document}
\maketitle

\textbf{Authors}

Juan Carlos Villaseñor-Derbez\(^{1, 2}\), C. Gabriela
Montaño-Moctezuma\(^3\), Guillermo Torres-Moye\(^1\), Antonio
Trujillo-Ortiz\(^1\), Arturo Ramírez-Valdez\(^{1, 4}\)

\textbf{Adscriptions}

\begin{enumerate}
\def\labelenumi{\arabic{enumi}.}
\item
  Facultad de Ciencias Marinas, Universidad Autónoma de Baja California,
  Km. 103 Carretera Tijuana-Ensenada, Ensenada, Baja California, México
  C.P. 22860
\item
  Bren School of Environmental Science \& Management, University of
  California Santa Barbara, Santa Barbara, CA
\item
  Instituto de Investigaciones Oceanológicas, Universidad Autónoma de
  Baja California, Km. 103 Carretera Tijuana-Ensenada, Ensenada, Baja
  California, México C.P. 22860
\item
  Scripps Institution of Oceanography, University of California San
  Diego, La Jolla, CA, USA
\end{enumerate}

\textbf{Corresponding Author}

Juan Carlos Villaseñor-Derbez, (+1) 207 205 8435,
\href{mailto:jvillasenor@bren.ucsb.edu}{\nolinkurl{jvillasenor@bren.ucsb.edu}}

\clearpage

\section{Abstract}\label{abstract}

\section{Key words}\label{key-words}

\clearpage

\section{Introduction}\label{introduction}

\begin{itemize}
\item
  Tallas son usadas frecuentemente en ciencia pesquera
\item
  Tallas minimas de captura
\item
  Calcular biomasa
\item
  Modelos estructurados por tallas / edades
\item
  Tallas en ecologia
\item
  Estructura de tallas puede indicar zonas de reclutamiento,
  agregaciones reproductivas
\item
  Determinar el estatus de establecimiento de una poblacion
\item
  Indicar diferencias en productividad (Jonno Wilson)
\item
  Size is a determinant of age and fecundity and egg quality
\end{itemize}

Kelp forests along the Eastern Pacific coast are seen as transboundary
resources (Torres-Moye 2013), and are subject to different regulations
on either side of the U.S. - Mexico border (Ramírez-Valdez et al. 2017).
Furthermore, resources associated to kelp forests on Mexican coasts have
received little attention when compared to their homologous across the
border (Ramírez-Valdez et al. 2017).

Thus, the only available information about size structures for fish
populations comes from landing reports that do not necessarily reflect
the composition and state of populations as the indicators (e.g.~size
structure) may be biased by the interests of fisheries operating in each
region (Essington et al., 2006). This lack of information impairs
decision making and disables the implementation of correct management
strategies.

Given the importance of size structures of fish populations, we discribe
size structures of the most abundant fish species along Baja California.

\textbf{Specifically, we compared size structures, fish size, and
proportion of potentially mature organisms of fish populations under
different degrees of anthropogenic pressure. This study represents the
first approach to comprehensively assess fish size-structures and
densities along the Mexican kelp forests through visual censuses and
that did not rely on landing data.}

\section{Methods}\label{methods}

\subsection{Area of study}\label{area-of-study}

The present study took place in the kelp forests off the coasts of Baja
California, Mexico (Fig. 1).

Fishing has been recognized as one of the most important sources of
pressure exerted on kelp forests (Shroeder \& Love, 2002; Coleman et
al., 2003). For the area of study, an average sport-fishing trip may
yield a catch of 19 organisms, equivalent to 54 Kg (Sosa-Nishizaki et
al., 2013).

Human population is unequally distributed along Baja California. The
larger cities on the Pacific coast of the state (Tijuana and Ensenada)
are located in the north, and there are vast unpopulated areas towards
the south. Fishing pressure (commercial and artisanal) is
heterogeneously distributed, with some major fishing grounds in front of
the large cities, but also some important areas south, in small towns.
Nevertheless, due to its proximity to the United States, and mainland
Mexico, there is also a higher availability of sport fishing activities
in the northern region, particularly in Ensenada and San Quintín
(Sosa-Nishizaki et al., 2013). Thus, we believe that an uneven
distribution of anthropogenic pressure exists in the Pacific coast of
Baja California, where the densely populated northern regions induce a
higher stress on the marine environment than small settlements in the
southern part of the state.

\includegraphics{Manuscript_files/figure-latex/unnamed-chunk-2-1.pdf}

\includegraphics{Manuscript_files/figure-latex/unnamed-chunk-4-1.pdf}

\subsection{Sampling}\label{sampling}

During October and November 2013 we visited 13 locations distributed
along 291 Km of coastline off the Pacific coasts of Baja California,
Mexico (Fig. 1, Table I). Trained scuba divers assessed fish communities
through visual censuses along 30 × 2 m belt transects. On each transect,
we registered species richness, abundances, and total lengths, in 5 cm
intervals. We performed a total of 133 belt transects, accounting for a
total surveyed area of 7,980 m\textsuperscript{2}. Individual divers
were randomized between survey groups for each sampling event in order
to reduce bias (Sandin et al., 2008).

\begin{table}[!h]
\centering
\begin{tabular}{lrrrr}
\toprule
Location & Latitude & Longitude & Bottom & Midwater\\
\midrule
ASA & 29.73211 & -115.7756 & 6 & 5\\
BMA & 32.01675 & -116.8831 & 6 & 5\\
ERE & 31.28125 & -116.4037 & 6 & 6\\
ERO & 29.90300 & -115.7762 & 5 & 1\\
ISJE & 29.78662 & -115.7990 & 6 & 6\\
\addlinespace
ISJP & 29.78457 & -115.7898 & 6 & 6\\
ISME & 30.49803 & -116.1191 & 6 & 6\\
ISMP & 30.48050 & -116.1170 & 6 & 5\\
ITSE & 31.79726 & -116.8000 & 6 & 6\\
ITSP & 31.80850 & -116.7987 & 6 & 6\\
\addlinespace
RET & 31.60444 & -116.6729 & 6 & 6\\
SMI & 31.91014 & -116.7581 & 6 & 6\\
SSI & 31.98244 & -116.8427 & 6 & 6\\
\bottomrule
\end{tabular}
\end{table}

\subsection{Data analysis}\label{data-analysis}

Id abundant species w IVB

\includegraphics{Manuscript_files/figure-latex/unnamed-chunk-8-1.pdf}

\[TL_{i,j} = \beta_{0i}Species_i + \beta_1 Latitude_j + \beta_2 Latitude_j \times Fished_i + \epsilon\]

\[TL_{i,j} = \beta_{0i}Species_i + \beta_1 Latitude_j + \beta_2 Latitude_j \times Fished_i + \beta_3Kelp_j + \epsilon\]

\[TL_{i,j} = \beta_{0i}Species_i + \beta_1 Latitude_j + \beta_2 Latitude_j \times Fished_i + \beta_3Habitat_j + \epsilon\]

\[TL_{i,j} = \beta_{0i}Species_i + \beta_1 Latitude_j + \beta_2 Latitude_j \times Fished_i + \beta_3Temp_j + \epsilon\]

\[TL_{i,j} = \beta_{0i}Species_i + \beta_1 Latitude_j + \beta_2 Latitude_j \times Fished_i + \beta_3Kelp_j + \beta_4Habitat_j + \beta_5Temp_j + \epsilon\]

\begin{itemize}
\item
  \(i\) = Species
\item
  \(j\) = Transect?
\end{itemize}

\clearpage

\section{Results}\label{results}

\begin{table}

\caption{\label{tab:unnamed-chunk-10}Biological value index for 16 fish species along the Kelp Forests of Baja California. BVI = Biological Value Index, rBVI = Relative Biological Value Index, cBVI = Cummulative Biological Value Index.}
\centering
\begin{tabular}[t]{>{\em}lrrr}
\toprule
Species & BVI & rBVI & cBVI\\
\midrule
Oxyjulis californica & 190 & 12.02 & 12.02\\
Chromis punctipinnis & 172 & 10.88 & 22.90\\
Semicossyphus pulcher & 153 & 9.68 & 32.57\\
Embiotoca jacksoni & 109 & 6.89 & 39.47\\
Hypsypops rubicundus & 104 & 6.58 & 46.05\\
\addlinespace
Paralabrax clathratus & 94 & 5.95 & 51.99\\
Rhacochilus vacca & 80 & 5.06 & 57.05\\
Sebastes atrovirens & 72 & 4.55 & 61.61\\
Hypsurus caryi & 62 & 3.92 & 65.53\\
Brachyistius frenatus & 59 & 3.73 & 69.26\\
\addlinespace
Oxylebius pictus & 52 & 3.29 & 72.55\\
Rhinogobiops nicholsii & 52 & 3.29 & 75.84\\
Girella nigricans & 46 & 2.91 & 78.75\\
Embiotoca lateralis & 35 & 2.21 & 80.96\\
Sebastes serranoides & 31 & 1.96 & 82.92\\
\addlinespace
Others & 270 & 17.08 & 100.00\\
Total & 1581 & 100.00 & NA\\
\bottomrule
\end{tabular}
\end{table}

\begin{figure}
\centering
\includegraphics{Manuscript_files/figure-latex/unnamed-chunk-11-1.pdf}
\caption{Mean total length by transect as a function of latitude. Marker
color indicates habitat (B = bottom, M = midwater), marker size
indicates Kelp Density (Fronds / m2).}
\end{figure}

\begin{landscape}


\begin{table}[!htbp] \centering 
  \caption{} 
  \label{} 
\tiny 
\begin{tabular}{@{\extracolsep{5pt}}lcccccc} 
\\[-1.8ex]\hline 
\hline \\[-1.8ex] 
\\[-1.8ex] & \multicolumn{6}{c}{Length (cm)} \\ 
 & Model 1 & Model 2 & Model 3 & Model 4 & Model5 & Model6 \\ 
\hline \\[-1.8ex] 
 C. punctipinnis & 32.562$^{***}$ (1.482) & 33.206$^{***}$ (1.497) & 29.665$^{***}$ (1.675) & 60.061$^{***}$ (5.615) & 71.227$^{***}$ (5.915) & 63.323$^{***}$ (5.846) \\ 
  E. jacksoni & 44.461$^{***}$ (1.541) & 45.081$^{***}$ (1.554) & 41.311$^{***}$ (1.756) & 72.044$^{***}$ (5.631) & 83.089$^{***}$ (5.925) & 74.965$^{***}$ (5.851) \\ 
  H. rubicundus & 116.228$^{***}$ (12.145) & 117.284$^{***}$ (12.200) & 116.178$^{***}$ (12.155) & 143.173$^{***}$ (13.377) & 83.653$^{***}$ (5.922) & 150.472$^{***}$ (13.475) \\ 
  O. californica & 32.514$^{***}$ (1.509) & 33.167$^{***}$ (1.526) & 29.610$^{***}$ (1.703) & 60.079$^{***}$ (5.637) & 71.391$^{***}$ (5.938) & 63.352$^{***}$ (5.866) \\ 
  P. clathratus & 119.770$^{***}$ (12.102) & 120.842$^{***}$ (12.159) & 119.841$^{***}$ (12.113) & 146.781$^{***}$ (13.365) & 88.126$^{***}$ (6.017) & 154.273$^{***}$ (13.461) \\ 
  R. vacca & 44.946$^{***}$ (1.626) & 45.559$^{***}$ (1.637) & 41.844$^{***}$ (1.830) & 72.449$^{***}$ (5.647) & 83.542$^{***}$ (5.940) & 75.411$^{***}$ (5.868) \\ 
  S. atrovirens & 117.849$^{***}$ (12.282) & 118.899$^{***}$ (12.336) & 117.885$^{***}$ (12.291) & 144.877$^{***}$ (13.519) & 84.567$^{***}$ (5.998) & 152.310$^{***}$ (13.623) \\ 
  S. pulcher & 120.814$^{***}$ (12.086) & 121.891$^{***}$ (12.139) & 120.768$^{***}$ (12.096) & 147.779$^{***}$ (13.323) & 88.802$^{***}$ (5.952) & 155.102$^{***}$ (13.428) \\ 
  Latitude & $-$0.788$^{***}$ (0.048) & $-$0.814$^{***}$ (0.049) & $-$0.687$^{***}$ (0.055) & $-$0.945$^{***}$ (0.057) & $-$1.141$^{***}$ (0.071) & $-$0.863$^{***}$ (0.059) \\ 
  Kelp &  & 0.331$^{***}$ (0.124) &  &  & 0.180 (0.130) & 0.226$^{*}$ (0.127) \\ 
  Habitat &  &  & $-$0.422$^{***}$ (0.094) &  & $-$0.376$^{***}$ (0.105) & $-$0.578$^{***}$ (0.100) \\ 
  Temperature &  &  &  & $-$1.291$^{***}$ (0.255) & $-$1.576$^{***}$ (0.279) & $-$1.609$^{***}$ (0.280) \\ 
  Latitude*Fished & $-$2.284$^{***}$ (0.392) & $-$2.299$^{***}$ (0.393) & $-$2.383$^{***}$ (0.393) & $-$2.266$^{***}$ (0.392) &  & $-$2.407$^{***}$ (0.394) \\ 
 \hline \\[-1.8ex] 
F & 6,496.3*** (df = 9; 14,688) & 5,879.6*** (df = 10; 14,687) & 5,882.0*** (df = 10; 14,687) & 5,866.6*** (df = 10; 14,687) & 5,409.8*** (df = 11; 14,686) & 4,967.0*** (df = 12; 14,685) \\ 
AIC & 92,012.55 & 92,006.44 & 91,999.16 & 91,996.38 & 92,140.90 & 91,969.28 \\ 
BIC & 92,096.10 & 92,097.59 & 92,090.31 & 92,087.52 & 92,239.64 & 92,075.61 \\ 
Observations & 14,698 & 14,698 & 14,698 & 14,698 & 14,698 & 14,698 \\ 
R$^{2}$ & 0.804 & 0.804 & 0.804 & 0.804 & 0.802 & 0.805 \\ 
\hline 
\hline \\[-1.8ex] 
\textit{Note:}  & \multicolumn{6}{r}{$^{*}$p$<$0.1; $^{**}$p$<$0.05; $^{***}$p$<$0.01} \\ 
\end{tabular} 
\end{table} 

\end{landscape}

\clearpage

\section{Discusion and Conclusion}\label{discusion-and-conclusion}

Removing organisms of specific sizes (\emph{i. e.} size-selective
harvesting) is a common practice in marine and terrestrial ecosystems
(Fenberg \& Roy, 2008). Commercial fisheries often focus their efforts
on large--sized organisms within a same species which yields a greater
revenue--effort ratio (Hepell et al., 2005; Hamilton et al., 2007).
Trophy--fishing follows the same pattern; larger fish represent a
greater challenge and thus produce more satisfaction when captured
(Shroeder \& Love, 2002). Size--selective harvesting can also be
market--driven when buyers have a preferred size, typically medium-sized
or ``dish-sized'' fish that can be better allocated in the market (Reddy
et al., 2013 / Aburto).

Fishing pressure (especially when size-selective) can modify life
histories of fishes (McBride et al., 2013). For example, Thompson \&
Stokes (1996) demonstrated how different fishing methods targeting
different sizes produced different size structures on Atlantic cod
(\emph{Gadus morhua}) populations by non-randomly distributing
mortality.

\clearpage

\section{References}\label{references}

\hypertarget{refs}{}
\hypertarget{ref-ramrezvaldez_2017-RH}{}
Ramírez-Valdez, Arturo, Octavio Aburto-Oropeza, Arafeh D Nur, Rodrigo
Beas-Luna, Jennifer E Caselle, Max C. N Castorani, Kyle Cavanaugh, et
al. 2017. ``Mexico-California Bi-National Initiative of Kelp Forest
Ecosystems and Fisheries.'' UC Office of the President: UC-Mexico
Initiative. Oakland, CA: University of California; eScholarship
University of California.
\url{http://escholarship.org/uc/item/8sp8j4xs}.

\hypertarget{ref-torresmoye_2013-mx}{}
Torres-Moye, G. 2013. ``Benthic Community Structure in Kelp Forests from
the Southern California Bight.'' \emph{Cienc Mar} 39 (3): 239--52.
doi:\href{https://doi.org/10.7773/cm.v39i3.2250}{10.7773/cm.v39i3.2250}.

\section{Other figures}\label{other-figures}

\includegraphics{Manuscript_files/figure-latex/unnamed-chunk-15-1.pdf}

\includegraphics{Manuscript_files/figure-latex/unnamed-chunk-16-1.pdf}

\includegraphics{Manuscript_files/figure-latex/unnamed-chunk-17-1.pdf}


\end{document}
